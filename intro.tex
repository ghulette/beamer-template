%%%%%%%%%%%%%%%%%%%%%%%%%%%%%%%%%%%%%%%%%%%%%%%%%%%%%%%%%%%%%%%%%%%%%%%%%%%%%%
\begin{frame}[fragile]{Introduction}
This talk is about encoding imperative programs in Coq. The goal is to be able 
to write programs like this:

\begin{block}{}
\begin{verbatim}
Definition factorial : com :=
  Z ::= AId X;
  Y ::= ANum 1;
  WHILE BNot (BEq (AId Z) (ANum 0)) DO
    Y ::= AMult (AId Y) (AId Z);
    Z ::= AMinus (AId Z) (ANum 1)
  END.
\end{verbatim}
\end{block}

The material has been mostly lifted from Benjamin Pierce's online textbook, 
Software Foundations.\footnote{\url{http://www.cis.upenn.edu/~bcpierce/sf/}}
\end{frame}

\begin{frame}[fragile]{Goals}
With these programs, our goals are to be able to:

\begin{itemize}
\item encode their formal syntax and semantics in Coq;
\item evaluate them under different initial conditions;
\item prove things about them (e.g., termination or some invariant condition);
\item prove \emph{meta-properties} about the language itself.
\end{itemize}
\end{frame}
