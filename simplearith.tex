%%%%%%%%%%%%%%%%%%%%%%%%%%%%%%%%%%%%%%%%%%%%%%%%%%%%%%%%%%%%%%%%%%%%%%%%%%%%%%
\section{Arithmetic Expressions}
\frame{\tableofcontents[currentsection]}

%%%%%%%%%%%%%%%%%%%%%%%%%%%%%%%%%%%%%%%%%%%%%%%%%%%%%%%%%%%%%%%%%%%%%%%%%%%%%%
\begin{frame}{Semantics: Simple Arithmetic}
Let's start with a smaller language: arithmetic expressions over natural
numbers.

\begin{itemize}
\item In the following ``big-step'' semantics, let $e$ range over expressions 
and $n$ over natural numbers.
\end{itemize}

\begin{block}{}
\[
\relrule{E\_ANum}{}{\evals{\co{ANum} n}{n}}
\]

\[
\relrule{E\_APlus}
  {\evals{e_1}{n_1} \qquad \evals{e_2}{n_2}}
  {\evals{\co{APlus} e_1 \, e_2}{n_1 + n_2}}
\]

\[
\relrule{E\_AMult}
  {\evals{e_1}{n_1} \qquad \evals{e_2}{n_2}}
  {\evals{\co{AMult} e_1 \, e_2}{n_1 \times n_2}}
\]
\end{block}
\end{frame}

%%%%%%%%%%%%%%%%%%%%%%%%%%%%%%%%%%%%%%%%%%%%%%%%%%%%%%%%%%%%%%%%%%%%%%%%%%%%%%
\begin{frame}[fragile]{Syntax}
We can encode the \emph{syntax} of this language in Coq as an inductive
datatype:

\begin{block}{}
\begin{verbatim}
Inductive aexp : Type := 
  | ANum   : nat -> aexp
  | APlus  : aexp -> aexp -> aexp
  | AMult  : aexp -> aexp -> aexp.
\end{verbatim}
\end{block}

and then use it to write expressions, like these:

\begin{block}{}
\begin{itemize}
  \item\texttt{ANum 5}
  \item\texttt{APlus (ANum 1) (ANum 2)}
  \item\texttt{AMult (APlus (ANum 1) (ANum 2)) (ANum 3)}
\end{itemize}
\end{block}
\end{frame}

%%%%%%%%%%%%%%%%%%%%%%%%%%%%%%%%%%%%%%%%%%%%%%%%%%%%%%%%%%%%%%%%%%%%%%%%%%%%%%
\begin{frame}[fragile]{Evaluation function}
\emph{Evaluation} reduces an expression to a natural number, according to
the semantics given above.
\begin{itemize}
\item One way to do this in Coq is with a recursively-defined \texttt{Fixpoint}
function.
\item Coq requires that such functions be total.
\end{itemize}

\begin{block}{}
\small\begin{verbatim}
Fixpoint aeval (e : aexp) : nat :=
  match e with
  | ANum n       => n
  | APlus a1 a2  => (aeval a1) + (aeval a2)
  | AMult a1 a2  => (aeval a1) * (aeval a2)
  end.
\end{verbatim}
\end{block}
\end{frame}

%%%%%%%%%%%%%%%%%%%%%%%%%%%%%%%%%%%%%%%%%%%%%%%%%%%%%%%%%%%%%%%%%%%%%%%%%%%%%%
\begin{frame}[fragile]{Evaluating expressions}
With \texttt{aeval}, we can force Coq to do some computation for us.

\begin{block}{}
\small\begin{verbatim}
Coq < Eval simpl in aeval (APlus (ANum 1) (ANum 2)).
     = 3 : nat
\end{verbatim}
\end{block}

\begin{block}{}
\small\begin{verbatim}
Coq < Eval simpl in aeval 
(AMult (APlus (ANum 1) (ANum 2)) (ANum 3)).
     = 9 : nat
\end{verbatim}
\end{block}
\end{frame}

%%%%%%%%%%%%%%%%%%%%%%%%%%%%%%%%%%%%%%%%%%%%%%%%%%%%%%%%%%%%%%%%%%%%%%%%%%%%%%
\begin{frame}[fragile]{Proofs}
We can also prove things about expressions.

\begin{itemize}
\item For example, that the following optimization is sound.
\end{itemize}

\begin{block}{}
\small{\begin{verbatim}
Fixpoint optimize_0plus (e:aexp) : aexp := 
  match e with
  | ANum n => ANum n
  | APlus (ANum 0) e2 => optimize_0plus e2
  | APlus e1 e2 => 
      APlus (optimize_0plus e1) (optimize_0plus e2)
  | AMult e1 e2 => 
      AMult (optimize_0plus e1) (optimize_0plus e2)
  end.
\end{verbatim}}
\end{block}

\begin{block}{}
\small{\begin{verbatim}
Theorem optimize_0plus_sound: forall e,
  aeval (optimize_0plus e) = aeval e.
Proof...
\end{verbatim}}
\end{block}
\end{frame}

%%%%%%%%%%%%%%%%%%%%%%%%%%%%%%%%%%%%%%%%%%%%%%%%%%%%%%%%%%%%%%%%%%%%%%%%%%%%%%
\begin{frame}[fragile]{Evaluation relation}
Another way to encode semantics is with a \textbf{relation} between 
expressions and their values.

\begin{itemize}
\item Relations need not be total.
\item More flexible for semantics with ``stuck states'' and/or potential 
non-termination.
\end{itemize}

\begin{block}{}
\small\begin{verbatim}
Inductive aevalR : aexp -> nat -> Prop :=
  | E_ANum : forall (n:nat), 
      (ANum n) || n
  | E_APlus : forall (e1 e2: aexp) (n1 n2 : nat), 
      (e1 || n1) -> (e2 || n2) -> 
      (APlus e1 e2) || (n1 + n2) 
  | E_AMult :  forall (e1 e2: aexp) (n1 n2 : nat), 
      (e1 || n1) -> (e2 || n2) -> 
      (AMult e1 e2) || (n1 * n2)
  where "e '||' n" := (aevalR e n).
\end{verbatim}
\end{block}
\end{frame}

%%%%%%%%%%%%%%%%%%%%%%%%%%%%%%%%%%%%%%%%%%%%%%%%%%%%%%%%%%%%%%%%%%%%%%%%%%%%%%
\begin{frame}[fragile]{Computing with relations}
With a relation, however, Coq can no longer do computation for us.

\begin{itemize}
\item Instead, we must provide a \emph{proof} that an expression will evaluate 
to a certain value.
\end{itemize}
\begin{block}{}
\small\begin{verbatim}
Example aevalR_ex1 : APlus (ANum 1) (ANum 2) || 3.
Proof.
  apply (E_APlus (ANum 1) (ANum 2) 1 2).
    apply E_ANum.
    apply E_ANum.
Qed.
\end{verbatim}
\end{block}
\end{frame}

%%%%%%%%%%%%%%%%%%%%%%%%%%%%%%%%%%%%%%%%%%%%%%%%%%%%%%%%%%%%%%%%%%%%%%%%%%%%%%
\begin{frame}[fragile]{Proofs with relations}
With a relation, we can still prove the same kinds of things.

\begin{itemize}
\item We prove them in a slightly different way.
\item The choice between functions and relations is, to some extent, aesthetic.
\end{itemize}

\begin{block}{}
\small\begin{verbatim}
Theorem optimize_sound_R: forall e n,
 e || n -> (optimize_0plus e) || n.
Proof...
\end{verbatim}
\end{block}
\end{frame}

%%%%%%%%%%%%%%%%%%%%%%%%%%%%%%%%%%%%%%%%%%%%%%%%%%%%%%%%%%%%%%%%%%%%%%%%%%%%%%
\begin{frame}[fragile]{Equivalence of functional/relational semantics}
We can prove that our two encodings are equivalent.

\begin{block}{}
\small\begin{verbatim}
Theorem aeval_iff_aevalR : forall a n,
  (a || n) <-> aeval a = n.
Proof.
  split.
    intros H; induction H; subst; reflexivity. 

    generalize dependent n.
    induction a; simpl; intros; subst; constructor;
       try apply IHa1; try apply IHa2; reflexivity. 
Qed.
\end{verbatim}
\end{block}
\end{frame}
